\documentclass[a4paper, 10pt]{article}

\usepackage[utf8]{inputenc}
\usepackage[brazilian]{babel}

% The following packages can be found on http:\\www.ctan.org
\usepackage{graphics} % for pdf, bitmapped graphics files
\usepackage{epsfig} % for postscript graphics files
\usepackage{mathptmx} % assumes new font selection scheme installed
\usepackage{times} % assumes new font selection scheme installed
\usepackage{amsmath} % assumes amsmath package installed
\usepackage{amssymb}  % assumes amsmath package installed

\title{\LARGE \bf
O Princípio da Indução e suas Aplicações
}

\author{Flávio L. C. de Moura}

\begin{document}
\maketitle

\begin{abstract}

Neste trabalho mostraremos diversas utilizações do princípio de indução em Ciência da Computação, por meio da solução de dois probelmas propostos.

\end{abstract}

\section{Introdução}

O trabalho será divido em duas partes, que serão respectivamente a solução do primeiro problema e a solução do segundo problema. 

\section{Problema 1:}

Prove a equivalência entre os princípios da indução forte (PIF) e da indução matemática (PIM).



\section{Problema 2:}

Provar a correção do algoritmo de ordenação de listas conhecido como \textit {insertion sort}, ou ordenação por inserção.
O pseudocódigo deste algoritmo é dado a seguir:\\

$\textit{InsertionSort(l)}=\begin{cases}
l,& \text{se l = []},\\
\textit{Insert(h,InsertionSort(l`))},& \text{se l = h :: l`}
\end{cases}$

$\textit{Insert(x,l)}=\begin{cases}
x :: [],& \text{se l = []},\\
x ::l,& \text{se l = h :: l` e $x \leq h$}
\end{cases}$


\end{document}
