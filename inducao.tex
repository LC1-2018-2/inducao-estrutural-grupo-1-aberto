\documentclass[a4paper, 10pt]{article}

\usepackage[utf8]{inputenc}
\usepackage[brazilian]{babel}

% The following packages can be found on http:\\www.ctan.org
\usepackage{graphics} % for pdf, bitmapped graphics files
\usepackage{epsfig} % for postscript graphics files
\usepackage{mathptmx} % assumes new font selection scheme installed
\usepackage{times} % assumes new font selection scheme installed
\usepackage{amsmath} % assumes amsmath package installed
\usepackage{amssymb}  % assumes amsmath package installed

\title{\LARGE \bf
O Princípio da Indução e suas Aplicações
}

\author{Leonardo R. do Nascimento\\ Eduardo F. Fernandes}

\begin{document}
\maketitle

\begin{abstract}

Neste trabalho mostraremos diversas utilizações do princípio de indução em Ciência da Computação, por meio da solução de dois probelmas propostos.

\end{abstract}

\section{Introdução}

O trabalho será divido em tres partes, que serão respectivamente a solução dos exercícios, do primeiro problema e a solução do segundo problema. 

\section{Exercícios}
1.Prove: \\
a) $n^2 < 2^n$, para $n \geq 5$ \\
\textbf{Prova por indução:} \\
a.\textbf{Base da indução:} a propriedade tem que ser verdadeira para o primeiro número da condição($n \geq 5$), que é o 5. \\

$P(5) = 5^2 < 2^5 \Longrightarrow{} 25 < 32$, que é trivialmente verdadeiro.\checkmark \\

\noindent A partir disso iremos para o proximo passo da nossa prova, o \textbf{passo indutivo}, nele assumimos a \textbf{hipótese indutiva}($P(n)$ como verdadeira, a fim de provar P(n+1).Desta maneira: \\

$P(n+1): (n+1)^2 < 2^{(n+1)} \Longrightarrow n^2+2n+1 < 2*2^n \Longrightarrow n^2+2n+1<2^n + 2^n$ \\ \\
Por hipotese de indução: $n^2 < 2^n \Longrightarrow n^2 + \textbf{2n+1} < 2^n + \textbf{2n+1}$. \\
Por tanto: consideramos substituir $n^2+2n+1$ por $2^n+2n+1$.  \\
Obtemos: $2^n+2n+1 < 2^n+2^n \Longleftrightarrow 2n+1 < 2^n$.  \\
Por isso  $n^2+2n+1  < 2^{n+1}$ \textbf{pois} $2^n+2n+1  < 2^{n+1} $,provando que a propriedade vale para P(n+1).\checkmark\\ 

\newpage

\noindent b) $n! > 2n$, para $n\geq 4$
{Prova por indução:} \\

\noindent a. {Base da indução:} a propriedade tem que ser verdadeira para o primeiro número da condição($n\geq4$), que é o 5. \checkmark\\
$P(4) = 4! > 2*4 \Longleftrightarrow 2*4>8$
A partir disso iremos para o proximo passo da nossa prova, o {passo indutivo}, nele assumimos a {hipótese indutiva}(P(n) como verdadeira, a fim de provar P(n+1).\\
Desta maneira: $P(n+1) = (n+1)!>2(n+1)$ \\ 
$=n!(n+1)>2(n+1)$ já que: $n! > 2$ para $n \geq 4$ \\
A prova é verdadeira.\checkmark\\ 
\\

\noindent c)  $\forall n \in \mathbb{N}$, $(n^3-n)$ é divisivel por 3.  \\
$(n^3 - n) = \mathbb{N}$.\\

\noindent a. Base indutiva: \\
$P(0) = (0^3 - 0)/3 = \mathbb{N}$\checkmark\\

\noindent b.Passo indutivo: Hipotese de indução:P(n) =$(n^3 - n)/3 = \mathbb{N}$ , logo P(n+1) = T? \\

$P(n+1) = [(n+1)^3 - (n+1)]/3 = \mathbb{N}$ \\

$P(n+1) = [n^3 - 3n^2 + 3n + 1 - (n+1)]/3 = \mathbb{N}$ \\

$P(n+1) = (n^3 - n + 3n^2 + 3n)/3 = \mathbb{N}$ \\

$P(n+1) =(n^3 - n)/3 + n^2 + n = \mathbb{N}$ \\

\noindent Ora, $P(n) =(n^3 - n)/3$ é nossa Hipotese de indução e sabemos que ela é $\mathbb{N}$, $(n^2 + n) $ tambem resulta em um $\mathbb{N}$, portanto soma de $\mathbb{N}$ resulta em $\mathbb{N}$. \\
A prova é verdadeira. \checkmark\\
\\ 


\noindent d)"A Soma dos primeiros n primeiros numeros impares = $2n-1$" \\
 $1+3+...+(2n-1)$ é uma PA de $r=2$. \\
 
 Soma PA = $(a_1 + a_n)n/2$ \\
 
\noindent a. Base da indução: $P(1)= (1+1)1/2 = 1 = (2*1) - 1$ \checkmark \\

\noindent b.Passo indutivo: Hipotese de indução: $P(n) = (a_1 + a_n)n/2 $ \\

$P(n+1) = {1+[1+((n+1)-1)2](n+1)} = (n+1)^2$ \\

$P(n+1) = [1+2n+1](n+1)/2 = (n+1)^2$ \\

$P(n+1) = 2[n+1](n+1)/2 = (n+1)^2$ \\

$P(n+1) = (n+1)^2 = (n+1^2)$  \checkmark 



\noindent f)\\ 
            $a_0 := 0$ \\
            $a_1 := 1$ \\
            $a_{i+2} := 2a_{i+1} - a_i$\\
            
            Hipotese de indução:\\
                                $a_n = 0, n=0$\\ 
                                $a_n = 1, n=1$\\
                                $a_n = 2(n-1)-(n-2)$, se $n\geq2$\\
    base da indução:\\
                    $a_1$ e $a_2$ \checkmark \\
                    \\
                    $a_2 = 2*(2-1)-(2-2) = 2 = 2a_1(2) - a_0(0)$\checkmark\\
                    \\
    passo indutivo: \\
                    $P(n+1) = 2(n+1-1)-(n+1-2) = 2a_n - a_{n-1}$\\
                    $2n - (n-1) = 2[2(n-1)-(n-2)] - [2(n-1-1)-(n-1-2)]$\\
                    $2n - (n-1) = 2[(2n-2)-(n-2)] - [2(n-2)-(n-3)]$\\
                    $2n-n+1 = 2(n)-[2n-4-n+3]$\\
                    $n+1 = 2n - (n-1)$\\
                    $n+1 = n+1$\checkmark\\
    
\noindent 2)"Todo numero natural $\geq2$ pode ser escrito como produto de primos"\\
De fato não consegui usar o produtório como hipótese de indução.
proponho uma outra, nela H.I.: $a*b =$ a qualquer número natural $\geq2$, sendo a e b produtos de números primos ou propriamente números primos.[Entre 2,n]\\

\centering{
$1<a<n$\\
$1<b<n$\\}
\raggedright
base indutiva: $2 = 2$(A base indutiva é simplesmente o primeiro primo $\geq2$ que pode ser escrito sozinho.)\checkmark\\
passo indutivo: Existem duas situações para P(n+1).\\
Na primeira n+1 é primo e assim como caso base, $n+1 = n+1$\checkmark\\
Na segunda situação n+1 não é primo, portanto\\

$n+1 = a*b$ ,[ $1<a<(n+1)$ ; $1<b<(n+1)$], pois n+1 não é primo então tem mais algum divisor diferente dele e 1.\\
portanto a e b estão entre [2,n] e são produto de primos ou primos.
concluimos que n+1 = produto de primos(ou primo$^1$) * produto de primos($^1$) = produto de primos.\\
É aqui que a ideia de Indução Forte entra, o fato da hipotese de indução generalizar para todos os casos entre [2,n] nos permite 
aplicar a propriedade para n+1, a indução matemática neste caso não seria suficiente.

\newpage

\section{Problema 1:}

Prove a equivalência entre os princípios da indução forte (PIF) e da indução matemática (PIM).\bigskip

Para provar essa equivalência começo apontando que a literatura nos diz que a indução Forte é uma variação da indução matemática e utilizada onde esta não abrange algum desafio matemático. Ou seja se fosse representado em um diagrama de Venn a indução Forte seria um circulo maior e dentro deste estaria contido um circulo menor, a indução matemática, a relação seria sobre qual sistema abrange mais provas.\\


O caso base dos dois métodos é semelhante e trivialmente equivalentes.\checkmark\\

Passo indutivo: Uma propriedade P vale para todos numeros entre [1,n] e então vale para n+1(Como vimos no caso dos primos), se vale para todos incluindo n, então P(n) = Verdadeiro e por consequentemente P(n+1) = Também é verdadeiro.Logo são equivalentes, quando PIF prova um problema, dentro da hipotese está o PIM.\checkmark




\section{Problema 2:}

Provar a correção do algoritmo de ordenação de listas conhecido como \textit {insertion sort}, ou ordenação por inserção.
O pseudocódigo deste algoritmo é dado a seguir:\\

\begin{center}
$\textit{InsertionSort(l)}=\begin{cases}
l,& \text{se l = []},\\
\textit{Insert(h,InsertionSort(l`))},& \text{se l = h :: l`}
\end{cases}$
\end{center}
\begin{center}
$\textit{Insert(x,l)}=\begin{cases}
x :: [],& \text{se l = []},\\
x ::l,& \text{se l = h :: l` e $x \leq h$}
\end{cases}$
\end{center}


\textbf{Solução} : 
Para provar que o algoritmo \textit{InsertionSort(l)} ordena corretamente uma lista, primeiramente temos que definir uma função que verifica a ordenação de uma lista qualquer, e posteriormente verificar se essa função retorna verdade para qualquer lista l aplicada na função InsrtionSort(l) por meio de indução na estrutura de l.\\

\begin{center}
$\textit{isSorted(l)}=\begin{cases}
true,& \text{se l = [] ou se l = h::[]},\\
\textit{isSorted(l`)},& \text{se $h \leq car(l`)$}\\
false,& \text{else}
\end{cases}$
\end{center}

\textbf{Indução estrutural em l} :

Base da Indução (BI): \\
l = [] (lista vazia) :\\
\ \ \ isSorted(InsertionSort(l)) $\stackrel{\text{inst}}{=}$ isSorted(InsertionSort([])) $\stackrel{\text{def}}{=}$ isSorted([]) $\stackrel{\text{def}}{=}$ true \checkmark\\~\\

Passo Indutivo (PI): Provar que isSorted(InsertionSort(l)) = true\\
Hipótese de Indução (HI): isSorted(InsertionSort(l`)) = true\\~\\

l = h :: l` :\\
\ \ \ 
isSorted(InsertionSort(h :: l)) $\stackrel{\text{def}}{=}$ isSorted(Insert(h, \underline{InsertionSort(l`)}))\\~\\

Pela hipótese de indução sabemos que a lista sublinhada já está ordenada, nos resta avaliar a função Insert. Ficaremos então com dois casos possiveis\\~\\
\newpage

caso 1 : se l` = x :: xs e $h \leq x$ :\\
Pela definição de Insert, Insert(h, InsertionSort(l`))) retornará uma lista ordenada \checkmark\\~\\

caso 2 se l' = x :: xs e $h \geq x$ :\\
Nesse caso, onde h é maior que o primeiro elemento da lista l', não há, na definição da função Insert um tratamento para específico para este caso. Portanto temos um caso indefinido. Concluimos assim que o algoritmo de ordenação insertion sort é incompleto.


\end{document}
